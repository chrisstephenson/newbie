\documentclass[a4paper,10pt]{article}
\usepackage[utf8]{inputenc}
\usepackage{hyperref}
\usepackage{comment}
\usepackage{enumerate}
\usepackage{etoolbox}

%opening
\title{Bilgisayar Bilimlerine başlamak isteyen birine tavsiyeler}

\author{Chris Stephenson}

\begin{document}

\maketitle

\begin{abstract}
Bilgisayar Bilimlerinde ilerlemek isteyen birileri için çeşitli seviyelerine göre ne yapmalı sorusuna hep beraber cevap verelim. 
\end{abstract}

\section*{Pratik}
\begin{enumerate}
  \item GNU/Linux
    \begin{enumerate}[(a)]
      \item Bilgisayarına yükle 
      \item Hergün yapılan işler içi BASH script yaz
      \item BASH iyice öğren
    \end{enumerate}
  \item GitHub
    \begin{enumerate}[(a)]
      \item GitHub Sudent Pack (Öğrenciysen)
      \item GitHub hesabı
      \item GiThub'da proje aç
      \item git'i komut satırından kullan.
    \end{enumerate}
  \item Server işleri
    \begin{enumerate}[(a)]
      \item Ufak bir virtual server al (Student Pack'tan digital ocean) yoksa Linode \$5. Paran yoksa birkaç arakadaşla beraber. Komut satır kök erişimi olsun (webhosting olmasın)
      \item Uygun çağdaş bir http serveri kur (nginx lighthttp ve saire)
      \item kendi domain ismini al ve kullan (dot.tk bedava)
    \end{enumerate}
  \item \LaTeX 
    \begin{enumerate}[(a)]
      \item CV'ini yap
    \end{enumerate}
   \item Hadoop
   \item Docker/Kubernetes
   \item Tensor Flow
    
\end{enumerate}

\section*{Teori}
\begin{enumerate}
  \item Okuma
    \begin{enumerate}[(a)]
      \item HTDP 2e (web)
      \item SICP (web)
      \item PLAI 2e (web) Chris Stephenson Programming Languages videolarıyla beraber 
      \item Chris Okasaki ``Purely Functional Data Structures''
      \item İyi bir Haskell kitabı (TBD)
    \end{enumerate}
  \item Online dersler
      \begin{enumerate}[(a)]
      \item Coursera Hinton Neural Networks
      \item Coursera  Andrew Ng Neural Networks
      \item Eklenecek
    \end{enumerate}

\end{enumerate}


\end{document}
