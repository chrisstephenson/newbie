\documentclass[a4paper,10pt]{article}
\usepackage[utf8]{inputenc}
\usepackage{hyperref}
\usepackage{comment}
\usepackage{enumerate}
\usepackage{etoolbox}

%opening
\title{Bilgisayar Bilimlerine başlamak isteyen birine tavsiyeler}

\author{Chris Stephenson}

\begin{document}

\maketitle

\begin{abstract}
Bilgisayar Bilimlerinde ilerlemek isteyen birileri için çeşitli seviyelerine göre ne yapmalı sorusuna hep beraber cevap verelim. 
\end{abstract}

\section*{Pratik}
\begin{enumerate}
  \item GNU/Linux
    \begin{enumerate}[(a)]
      \item Bilgisayarına yükle 
      \item Hergün yapılan işler içi BASH script yaz
      \item BASH iyice öğren
    \end{enumerate}
  \item GitHub
    \begin{enumerate}[(a)]
      \item GitHub Sudent Pack (Öğrenciysen)
      \item GitHub hesabı
      \item GiThub'da proje aç
      \item git'i komut satırından kullan.
    \end{enumerate}
  \item Server işleri
    \begin{enumerate}[(a)]
      \item Ufak bir virtual server al (Student Pack'tan digital ocean) yoksa Linode \$5. Paran yoksa birkaç arakadaşla beraber. Komut satır kök erişimi olsun (webhosting olmasın)
      \item Uygun çağdaş bir http serveri kur (nginx lighthttp ve saire)
      \item kendi domain ismini al ve kullan (dot.tk bedava)
    \end{enumerate}
  \item \LaTeX 
    \begin{enumerate}[(a)]
      \item CV'ini yap
    \end{enumerate}
   \item Hadoop
   \item Docker/Kubernetes
   \item Tensor Flow
    
\end{enumerate}

\section*{Teori}
\begin{enumerate}
  \item Okuma
    \begin{enumerate}[(a)]
      \item HTDP 2e (web)
      \item SICP (web)
      \item PLAI 2e (web) Chris Stephenson Programming Languages videolarıyla beraber 
      \item Chris Okasaki ``Purely Functional Data Structures''
      \item İyi bir Haskell kitabı (TBD)
    \end{enumerate}
  \item Online dersler
      \begin{enumerate}[(a)]
      \item Coursera Hinton Neural Networks
      \item Coursera  Andrew Ng Neural Networks
      \item Eklenecek
    \end{enumerate}

\end{enumerate}


\section*{düzenlenecek metin}

Gelen soruya yazdığım bir cevap. Boş gitmesin diye buraya attım.....

Hazırlamakta olduğum taslağı github'ta bulmuşsun.

Önerdiğim dil seyahatı:
 

Kitap HtDP(2e), dil Racket
Kitap SICP, dil Racket
Kitap gönderdiğim haskell kitabı, dil Haskell
Kitap PLAI(2e), dil racket-plai

Matematiğe gelince.

Ali Nesin'in yeni ve eski kitapları (lise matematiği kitapları, önermeler mantığı kitabı) ve yeni çıkan "derin matematik" videoları.

Yetmedi:
Discrete mathematics (kitap gönderiyorum)
Group/field/ring theory
Lineer cebir (kitap çok, ben uzman değilim)

Yetmedi:
Bilgisayar Bilimleri teorisi
Önce Sipser Introduction to Computational theory
ondan sonra Barendregt Lambda calculus
onden sonra Category Theory (kitap gönderdim) videoları burada
https://www.youtube.com/watch?v=I8LbkfSSR58


Teşekkürler, HtDP/sicp nedir? Dr. Racket'e giriş için "BilgisayarKavramlari" YouTube kanalında az da olsa içerik buldum..
HtDp ve SICP iki kitap. İkisi de online. HtDP için ikinci baskısını (2e)  tercih et.
http://www.ccs.neu.edu/home/matthias/HtDP2e/
How to Design Programs, Second Edition
© 1 August 2014 MIT Press This material is copyrighted and provided under the Creative Commons CC BY-NC-ND license [interpretation].
ccs.neu.edu
https://mitpress.mit.edu/sicp/full-text/book/book.html
Structure and Interpretation of Computer Programs
mitpress.mit.edu
http://cs.brown.edu/courses/cs173/2012/book/
Programming Languages: Application and Interpretation
cs.brown.edu
HtDp/Racket için benin vimeo kanalımda Buşent arkadşının verdiği video ders de var. https://vimeo.com/album/1987192
Comp 149 - How to Design Programs on Vimeo




   

\end{document}
