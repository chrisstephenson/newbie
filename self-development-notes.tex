\documentclass[a4paper,10pt]{article}
\usepackage[utf8]{inputenc}
\usepackage{amssymb}
\usepackage{hyperref}


%opening
\title{Self Development Notes}
\author{Chris Stephenson}

\begin{document}

\maketitle
\begin{abstract}
Assistants at the Nesin Matematik Köyü 2024 \textit{Cebir ve Programcılık Atölyesi} asked me for some recommendations about how to develop themselves as Computer Scientists. I took the opportunity to collect together  and update information that is distributed across different places in \href{https://chrisstephenson.org}{my Computer Science education web site}.
 
\end{abstract}
\newpage
\section{Practical skills}

\subsection{Linux and server skills}
Set up your own linux server.

Any competent Computer Scientist should have the necessary skills.

For 4.50 euro a month you can rent a powerful virtual server from Contabo.

The CMPE 283 worksheets are a good guide to setting up a server, but out of date. They refer to an old version of Apache and you might prefer to use NGINX instead of Apache these days. The guide needs updating.  The guide refers to Digital Ocean which we used when I gave CMPE 283 at Bilgi. Instead of Digital Ocean use  contabo.com. It is cheaper at only 4.50 euro a month and the virtual servers have much more capacity.

You can do a lot with your server. You can set up your own private VPN. Buy a personal .name.tr domain for a few TL. Set up a basic wordpress personal blog. Set up your own email server with your personal address.

\subsection{\LaTeX}

Use \LaTeX to prepare your documents. They will look like proper scientific articles. A CV prepared in \LaTeX is a must for academia and a plus for anywhere else. \LaTeX is not very easy to use, but you can work miracles with it once you learn it. No-one knows all of \LaTeX. You laern as much as you need as you go. 

This note and the \textit{Cebir ve Programcılık Atölyesi Çalışma Defteri} were both made with \LaTeX.

There are many development environments for \LaTeX. I use Kile under Linux. There are others for Windows and MacOS.  

\subsection{Your own computer}
Preferably make your own computer dual boot with Linux, especially if you use Windows. If you do not want to do this, you can still use the Unix command line on a MacOS computer. Did you know that? MacOS has a perfectly usable command line that lets you do all kinds of things that are really hard (like just moving files from one place to another) from the MacOS user interface. You just need to give the terminal the necessary permissions from \textit{settings}. Find the command line and use it.  

\section{Programming and languages}

My old courses and their (online) text books:-

\subsection{Comp 313}
Book is SICP

\href{https://mitp-content-server.mit.edu/books/content/sectbyfn/books_pres_0/6515/sicp.zip/full-text/book/book.html}{Structure and Interpretation of Computer Programs - Full Text}
\subsection{Comp 314}
\href{https://www.plai.org/}{Programming Languages Application and Interpretation}
\section{Theory}

Sipser (ekte)

(A hard read. Read as much as you can) 

Barendregdt lambda calculus paper (ekte)
Barendregdt lambda calculus book


\section{Programming Culture}
Programming Pearls

Mythical Man month

Programming Style

\end{document}
